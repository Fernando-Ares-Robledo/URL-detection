Para el desarrollo de este trabajo se ha seguido el método ágil, el cual se caracteriza por su flexibilidad y adaptabilidad ante cambios y nuevas necesidades que puedan surgir a lo largo del proyecto. Este enfoque permite iteraciones rápidas y continuas, asegurando una mejora constante del producto final mediante ciclos de retroalimentación y desarrollo incremental.

Las posibles estrategias consideradas para llevar a cabo este trabajo incluyeron:

\begin{itemize}
    \item \textbf{Desarrollar un producto nuevo:} Esta estrategia implica crear un sistema completamente nuevo para la detección de \glspl{url} maliciosas desde cero, incluyendo el desarrollo de modelos de \textit{machine learning}, la implementación de un {\gls{plugin}} de navegador y un \gls{dashboard} web.
    \item \textbf{Adaptar un producto existente:} Esta estrategia se basaría en modificar y mejorar sistemas de detección de \glspl{url} maliciosas ya existentes, añadiendo nuevas características y funcionalidades para cumplir con los objetivos del trabajo.
\end{itemize}

La estrategia elegida es \textbf{desarrollar un producto nuevo}. Esta elección se debe a varias razones:

\begin{itemize}
    \item Permite un mayor control sobre el diseño y la implementación del sistema, asegurando que todas las funcionalidades necesarias estén integradas desde el principio.
    \item Facilita la incorporación de características innovadoras y específicas que pueden no estar presentes en productos existentes.
    \item Garantiza una mayor flexibilidad para adaptar el sistema a futuros requerimientos y mejoras, sin las limitaciones impuestas por una arquitectura previa.
    \item Alinea el desarrollo del sistema con los principios de sostenibilidad, ética y diversidad, asegurando que se cumplan los estándares más altos desde el inicio del proyecto.
\end{itemize}

