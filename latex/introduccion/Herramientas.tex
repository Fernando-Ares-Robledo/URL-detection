En este proyecto se utilizaron diversas herramientas, organizadas en diferentes grupos según su función. A continuación, se describen brevemente las principales herramientas utilizadas.

\subsection*{Desarrollo y Programación}
\begin{itemize}
    \item \textbf{Python}: Lenguaje de programación utilizado para el desarrollo del extractor de características, el modelo de machine learning y la integración con el plugin y el dashboard.
    \item \textbf{Visual Studio Code}: Editor de código fuente utilizado para escribir y depurar el código Python, HTML, JavaScript y otros lenguajes utilizados en el proyecto.
    \item \textbf{Flask}: Microframework de Python utilizado para el desarrollo del plugin de navegador, proporcionando una estructura sencilla para crear aplicaciones web.
    \item \textbf{JavaScript}: Lenguaje de programación utilizado en el desarrollo del plugin de navegador para manipular el DOM y manejar interacciones del usuario.
    \item \textbf{Streamlit}: Framework de Python utilizado para el desarrollo del dashboard interactivo, facilitando la creación de aplicaciones web para la visualización de datos.
\end{itemize}

\subsection*{Gestión de Proyecto y Colaboración}
\begin{itemize}
    \item \textbf{GitHub}: Plataforma de alojamiento de código fuente utilizada para el control de versiones y la colaboración en el desarrollo del proyecto.
    \item \textbf{Trello}: Herramienta de gestión de proyectos utilizada para planificar y organizar las tareas del proyecto, permitiendo un seguimiento visual del progreso.
\end{itemize}

\subsection*{Documentación y Presentación}
\begin{itemize}
    \item \textbf{Overleaf}: Plataforma de edición colaborativa de documentos LaTeX utilizada para la redacción y formateo de la memoria del proyecto.
    \item \textbf{Google Documents}: Herramienta de procesamiento de texto utilizada para la elaboración de documentos y la colaboración en tiempo real.
    \item \textbf{PowerPoint}: Herramienta de creación de presentaciones utilizada para preparar la defensa del proyecto ante el tribunal.
\end{itemize}

\subsection*{Análisis y Modelado}
\begin{itemize}
    \item \textbf{scikit-learn}: Biblioteca de Python utilizada para el entrenamiento y evaluación de modelos de machine learning, proporcionando una amplia gama de algoritmos y herramientas de modelado.
    \item \textbf{matplotlib}: Biblioteca de Python utilizada para la generación de gráficos y visualizaciones, facilitando el análisis exploratorio de datos y la presentación de resultados.
    \item \textbf{numpy}: Biblioteca de Python utilizada para la manipulación eficiente de arreglos y operaciones matemáticas, fundamental en el procesamiento de datos.
    \item \textbf{scipy}: Biblioteca de Python que complementa a numpy, proporcionando herramientas adicionales para la computación científica y el análisis de datos.
\end{itemize}

\subsection*{Servicios Web}
\begin{itemize}
    \item \textbf{Google Safe Browsing}: Servicio utilizado para la detección de URLs maliciosas, proporcionando una capa adicional de seguridad en el análisis de URLs.
    \item \textbf{PhishTank}: Plataforma colaborativa para la identificación de sitios de phishing, utilizada como fuente de datos para URLs maliciosas.
    \item \textbf{URLhaus}: Servicio que recopila y comparte información sobre URLs maliciosas, utilizado para obtener datos de entrenamiento y evaluación del modelo de machine learning.
\end{itemize}