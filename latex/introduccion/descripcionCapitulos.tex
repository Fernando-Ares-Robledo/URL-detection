
Este trabajo final de máster se estructura en varios capítulos, cada uno enfocado en distintos aspectos relacionados con los ataques de ransomware, su prevención, y medidas reactivas. A continuación, se presenta un breve resumen de cada capítulo:\medskip

\subsection*{Introducción}
Se establece el contexto y la justificación del estudio, incluyendo el punto de partida y las aportaciones realizadas. Se definen los objetivos generales, específicos, de sostenibilidad y ética, así como los objetivos de entrega del proyecto y los que debe cumplir el sistema implementado. Además, se aborda el impacto en sostenibilidad, ético-social y de diversidad; el enfoque y método seguido; la planificación del trabajo; el análisis de riesgo del proyecto; y las herramientas utilizadas.

\subsection*{Estado del Arte}
En esta sección se realiza una revisión exhaustiva de la literatura existente sobre la detección de URLs maliciosas. Se describen las técnicas utilizadas, como el análisis basado en listas negras, análisis heurístico, análisis de contenido y métodos de machine learning. También se analizan las herramientas y soluciones actualmente disponibles en el mercado, destacando sus características y limitaciones.

\subsection*{Metodología}
Se detalla el enfoque metodológico adoptado para el desarrollo del sistema de detección de URLs maliciosas. Esto incluye la descripción del proceso de extracción de características, la selección y entrenamiento de los modelos de machine learning, y la implementación del plugin y el dashboard. Además, se presentan las métricas de evaluación utilizadas para medir el rendimiento del sistema.

\subsection*{Resultados y Discusión}
Se presentan los resultados obtenidos a partir de la evaluación del sistema. Esto incluye los resultados de las pruebas realizadas con el plugin y el dashboard, así como el desempeño de los modelos de machine learning. Se discuten los hallazgos más relevantes, las posibles causas de errores y las implicaciones de los resultados obtenidos.

\subsection*{Conclusiones y Trabajos Futuros}
Se resumen las principales conclusiones del estudio, evaluando el cumplimiento de los objetivos planteados. Se destacan las limitaciones del sistema actual y se proponen líneas de trabajo futuro para mejorar la detección de URLs maliciosas, incluyendo la recopilación continua de datos, el desarrollo de modelos más avanzados y la colaboración con otras entidades.

\subsection*{Glosario}
Se proporciona una lista de términos y definiciones relevantes utilizados a lo largo del documento, con el objetivo de facilitar la comprensión del lector.

\subsection*{Bibliografía}
Se incluyen todas las referencias bibliográficas citadas en el documento, siguiendo el formato adecuado. Esta sección recoge las fuentes utilizadas para fundamentar el estudio y para la revisión del estado del arte.

\subsection*{Anexos}
Se presentan materiales adicionales que complementan el contenido principal del documento. Esto puede incluir códigos fuente, datos experimentales, detalles técnicos, gráficos adicionales y cualquier otra información relevante que apoye los hallazgos y conclusiones del estudio.