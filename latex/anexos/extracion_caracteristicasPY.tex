\documentclass{article}
\usepackage{amsmath}
\usepackage{hyperref}
\usepackage{listings}
\usepackage{color}


\usepackage{listings}

%========================================================================%$$
%                                                                        %$$
%                         Configuración tabla1                           %$$
%                                                                        %$$
%========================================================================%$$
%tabla sin lineas verticales                                             %$$
                                                                         %$$
% Texto en negrita para el encabezado                                    %$$
\renewcommand\theadfont{\bfseries}                                       %$$
 % Centra el texto en el encabezado (centrado vertical y horizontal)     %$$
\renewcommand\theadalign{cc}                                             %$$
 % Fondo del encabezado                                                  %$$
\renewcommand\theadset{\cellcolor{tableheadcolor}}                       %$$
\setlength{\arrayrulewidth}{0.4mm}                                       %$$
%color lineas                                                            %$$
\arrayrulecolor{LightBlue}                                               %$$
                                                                         %$$
%========================================================================%$$
%                                                                        %$$
%                  FIN de Configuración tabla1                           %$$
%                                                                        %$$
%========================================================================%$$




\definecolor{VIUnaranja}{HTML}{e75114}

\definecolor{UOCBlue}{RGB}{0,84,159}
\definecolor{LightBlue}{HTML}{e75114}
\definecolor{headercolor}{gray}{0.85}
\definecolor{tableHeader}{RGB}{211, 211, 211}
\definecolor{rowStripes}{RGB}{242, 242, 242}
%\definecolor{headercolor}{RGB}{79, 129, 189}
\definecolor{oddRow}{RGB}{233, 236, 239}
\definecolor{evenRow}{RGB}{255, 255, 255}
\definecolor{darkblue}{RGB}{0,0,102}
\definecolor{lightgray}{RGB}{240,240,240}

\tcbuselibrary{breakable, skins}

\definecolor{phasecolor}{RGB}{220,230,240} % Color de fondo para las fases
\definecolor{sprintcolor}{RGB}{210,220,230} % Color de fondo para los sprints


% Configuración de listings para código
\definecolor{codegray}{rgb}{0.5,0.5,0.5}
\definecolor{codeblue}{rgb}{0.0,0.0,0.6}
\definecolor{codegreen}{rgb}{0,0.6,0}

%
\lstdefinestyle{mystyle}{
    backgroundcolor=\color{white},   
    commentstyle=\color{codegreen},
    keywordstyle=\color{codeblue},
    numberstyle=\tiny\color{codegray},
    stringstyle=\color{codegreen},
    basicstyle=\ttfamily\footnotesize,
    breakatwhitespace=false,         
    breaklines=true,                 
    captionpos=b,                    
    keepspaces=true,                 
    numbers=left,                    
    numbersep=5pt,                  
    showspaces=false,                
    showstringspaces=false,
    showtabs=false,                  
    tabsize=2
}
\lstset{style=mystyle}

\lstset{style=mystyle}

\begin{document}

\section{Extracción de características de URLs}

\subsection{Código de la clase ExtractorCaracteristicasURL}

\begin{lstlisting}[language=Python, caption=Clase para extraer características de URLs]
import psycopg2
from urllib.parse import urlparse
import re
import whois
import ssl
import socket
import requests
from collections import Counter
import math
import geoip2.database
import datetime
from psycopg2 import sql
from IPython.display import clear_output

class ExtractorCaracteristicasURL:
    def __init__(self, db_config, geoip_db_path):
        self.conn = psycopg2.connect(**db_config)
        self.cur = self.conn.cursor()
        self.geoip_reader = geoip2.database.Reader(geoip_db_path)
        
        # Crear la tabla de características si no existe
        self.cur.execute("""
        CREATE TABLE IF NOT EXISTS CaracteristicasURL (
            id SERIAL PRIMARY KEY,
            url TEXT NOT NULL,
            longitud INTEGER,
            cantidad_digitos INTEGER,
            cantidad_letras INTEGER,
            count_punto INTEGER,
            count_guion INTEGER,
            count_guion_bajo INTEGER,
            count_slash INTEGER,
            count_interrogacion INTEGER,
            count_igual INTEGER,
            count_arroba INTEGER,
            count_ampersand INTEGER,
            count_exclamacion INTEGER,
            count_espacio INTEGER,
            count_tilde INTEGER,
            count_coma INTEGER,
            count_mas INTEGER,
            count_asterisco INTEGER,
            count_numeral INTEGER,
            count_dolar INTEGER,
            count_porcentaje INTEGER,
            domain TEXT,
            domain_length INTEGER,
            cantidad_vocales_dominio INTEGER,
            ip_en_dominio BOOLEAN,
            directory TEXT,
            directory_length INTEGER,
            file TEXT,
            file_length INTEGER,
            parameters TEXT,
            parameters_length INTEGER,
            tld TEXT,
            tld_present BOOLEAN,
            numero_parametros INTEGER,
            email_presente BOOLEAN,
            tls_version TEXT,
            tld_type TEXT,
            tld_manager TEXT,
            whois_registrar TEXT,
            whois_creation_date TIMESTAMP,
            whois_expiration_date TIMESTAMP,
            whois_updated_date TIMESTAMP,
            whois_status TEXT,
            es_acortada BOOLEAN,
            numero_subdominios INTEGER,
            entropia_sld FLOAT,
            ciudad TEXT,
            pais TEXT,
            dominio_benigno BOOLEAN,
            tiene_palabras_sospechosas BOOLEAN,
            palabras_detectadas TEXT,
            tiene_hexadecimal BOOLEAN,
            redirige BOOLEAN,
            esta_registrada BOOLEAN,
            esta_online BOOLEAN,
            tiene_directorios BOOLEAN,
            tiene_file BOOLEAN,
            edad_dominio INTEGER,
            tiempo_restante INTEGER,
            queries_buenas BOOLEAN,
            queries_malas BOOLEAN,
            maligna BOOLEAN
        )
        """)
        self.conn.commit()

    def contar_caracteres(self, texto):
        counter = Counter(texto)
        return counter

    def calcular_entropia(self, texto):
        probabilidad = [float(texto.count(c)) / len(texto) for c in dict.fromkeys(list(texto))]
        entropia = - sum([p * math.log(p) / math.log(2.0) for p in probabilidad])
        return entropia

    def obtener_ip_de_dominio(self, url):
        try:
            dominio = urlparse(url).netloc
            direccion_ip = socket.gethostbyname(dominio)
            return direccion_ip
        except socket.gaierror as e:
            print(f"Error al obtener la IP del dominio: {e}")
            return None

    def obtener_ubicacion_de_ip(self, ip):
        try:
            respuesta = self.geoip_reader.city(ip)
            ubicacion = {
                "ciudad": respuesta.city.name,
                "pais": respuesta.country.name
            }
            return ubicacion
        except Exception as e:
            print(f"Error al obtener la ubicación de la IP: {e}")
            return {'ciudad': None, 'pais': None}

    def verificar_dominio(self, dominio):
        self.cur.execute("SELECT benigno FROM dominios WHERE domain = %s OR domain = %s", (dominio, dominio.replace('www.', '')))
        result = self.cur.fetchone()
        return result[0] if result else False

    def tiene_palabras_sospechosas(self, url, dominio):
        palabras_sospechosas = set()
        self.cur.execute("SELECT palabra FROM palabras")
        palabras = [row[0] for row in self.cur.fetchall()]

        url_sin_protocolo = re.sub(r'^https?://', '', url)
        partes_url = re.split(r'\W+', url_sin_protocolo)
        partes_dominio = re.split(r'\W+', dominio)

        for palabra in palabras:
            if palabra in partes_url or palabra en partes_dominio:
                palabras_sospechosas.add(palabra)

        return bool(palabras_sospechosas), list(palabras_sospechosas)

    def es_hexadecimal(self, url):
        return bool(re.search(r'0x[0-9a-fA-F]+', url))

    def verificar_online_y_redirigida(self, url):
        try:
            respuesta = requests.head(url, allow_redirects=True, timeout=2)
            esta_online = respuesta.status_code == 200
            es_redirigida = respuesta.url != url
            return esta_online, es_redirigida
        except requests.RequestException as e:
            print(f"Error al comprobar la URL: {e}")
            return False, False

    def verificar_registro(self, dominio):
        try:
            w = whois.whois(dominio)
            return bool(w.creation_date)
        except Exception as e:
            return False

    def calcular_edad_dominio(self, creation_date, expiration_date):
        if creation_date and expiration_date:
            return (expiration_date - creation_date).days
        return 0

    def calcular_tiempo_restante(self, expiration_date):
        if expiration_date:
            return (expiration_date - datetime.datetime.now()).days
        return 0

    def extraer_query(self, url):
        parsed_url = urlparse(url)
        return parsed_url.path + '?' + parsed_url.query if parsed_url.query else parsed_url.path

    def comprobar_query_en_db(self, query):
        self.cur.execute("SELECT * FROM queries WHERE query = %s", (query,))
        resultado = self.cur.fetchone()
        return bool(resultado), resultado

    def buscar_en_db(self, query):
        existe, resultado = self.comprobar_query_en_db(query)
        if existe:
            return query, resultado

        parts = query.split('/')
        for i in range(1, len(parts)):
            subquery = '/' + '/'.join(parts[i:])
            existe, resultado = self.comprobar_query_en_db(subquery)
            if existe:
                return subquery, resultado

        for i in range(len(parts) - 1, 0, -1):
            subquery = '/'.join(parts[:i])
            existe, resultado = self.comprobar_query_en_db(subquery)
            if existe:
                return subquery, resultado

        return None, None

    def asegurar_http(self, url):
        if not url.startswith(('http://', 'https://')):
            return 'http://' + url
        return url

    def extraer_caracteristicas(self, url, maligna):
        url = self.asegurar_http(url)
        parsed_url = urlparse(url)
        dominio = parsed_url.netloc
        ruta = parsed_url.path
        archivo = parsed_url.path.split('/')[-1] if '/' in parsed_url.path else ''
        parametros = parsed_url.query

        url_counter = self.contar_caracteres(url)
        dominio_counter = self.contar_caracteres(dominio)
        ruta_counter = self.contar_caracteres(ruta)
        archivo_counter = self.contar_caracteres(archivo)
        parametros_counter = self.contar_caracteres(parametros)

        tld_info = self.verificar_tld(dominio.split('.')[-1] if '.' in dominio else '')

        whois_info = self.obtener_info_whois(dominio)

        sld = dominio.split('.')[-2] if '.' in dominio else dominio
        entropia_sld = self.calcular_entropia(sld)

        esta_online, redirige = self.verificar_online_y_redirigida(url)

        if esta_online:
            ip = self.obtener_ip_de_dominio(url)
            ubicacion = self.obtener_ubicacion_de_ip(ip) if ip else {'ciudad': None, 'pais': None}
        else:
            ip = None
            ubicacion = {'ciudad': None, 'pais': None}

        dominio_benigno = self.verificar_dominio(dominio)

        tiene_palabras_sospechosas, palabras_detectadas = self.tiene_palabras_sospechosas(url, dominio)

        tiene_hexadecimal = self.es_hexadecimal(url)

        esta_registrada = self.verificar_registro(dominio)

        edad_dominio = self.calcular_edad_dominio(whois_info['creation_date'], whois_info['expiration_date'])
        tiempo_restante = self.calcular_tiempo_restante(whois_info['expiration_date'])

        subdominios = dominio.split('.')[:-2] if len(dominio.split('.')) > 2 else []

        tiene_directorios = bool(parsed_url.path and parsed_url.path != '/')
        tiene_file = bool(archivo)

        query = self.extraer_query(url)
        subquery, resultado_query = self.buscar_en_db(query)

        queries_buenas = False
        queries_malas = False

        if subquery and resultado_query:
            queries_buenas = resultado_query[2]
            queries_malas = not resultado_query[2]

        caracteristicas = {
            'url': url,
            'longitud': len(url),
            'cantidad_digitos': sum(c.isdigit() for c in url),
            'cantidad_letras': sum(c.isalpha() for c in url),
            'count_punto': url_counter['.'],
            'count_guion': url_counter['-'],
            'count_guion_bajo': url_counter['_'],
            'count_slash': url_counter['/'],
            'count_interrogacion': url_counter['?'],
            'count_igual': url_counter['='],
            'count_arroba': url_counter['@'],
            'count_ampersand': url_counter['&'],
            'count_exclamacion': url_counter['!'],
            'count_espacio': url_counter[' '],
            'count_tilde': url_counter['~'],
            'count_coma': url_counter[','],
            'count_mas': url_counter['+'],
            'count_asterisco': url_counter['*'],
            'count_numeral': url_counter['#'],
            'count_dolar': url_counter['$'],
            'count_porcentaje': url_counter['%'],
            'domain': dominio,
            'domain_length': len(dominio),
            'cantidad_vocales_dominio': sum(c in 'aeiouAEIOU' for c in dominio),
            'ip_en_dominio': bool(ip),
            'directory': ruta,
            'directory_length': len(ruta),
            'file': archivo,
            'file_length': len(archivo),
            'parameters': parametros,
            'parameters_length': len(parametros),
            'tld': (dominio.split('.')[-1] if '.' in dominio else ''),
            'numero_parametros': len(parametros.split('&')) if parametros else 0,
            'email_presente': bool(re.search(r'\b[A-Z0-9._%+-]+@[A-Z0-9.-]+\.[A-Z]{2,}\b', url, re.I)),
            'tls_version': self.obtener_version_tls(url),
            'tld_type': tld_info['type'],
            'tld_manager': tld_info['tld_manager'],
            'whois_registrar': whois_info['registrar'],
            'whois_creation_date': whois_info['creation_date'],
            'whois_expiration_date': whois_info['expiration_date'],
            'whois_updated_date': whois_info['updated_date'],
            'whois_status': whois_info['status'],
            'es_acortada': self.es_url_acortada_combinada(url),
            'numero_subdominios': len(subdominios),
            'entropia_sld': entropia_sld,
            'ciudad': ubicacion['ciudad'],
            'pais': ubicacion['pais'],
            'dominio_benigno': dominio_benigno,
            'tiene_palabras_sospechosas': tiene_palabras_sospechosas,
            'palabras_detectadas': ', '.join(palabras_detectadas),
            'tiene_hexadecimal': tiene_hexadecimal,
            'redirige': redirige,
            'esta_registrada': esta_registrada,
            'esta_online': esta_online,
            'edad_dominio': edad_dominio,
            'tiempo_restante': tiempo_restante,
            'tiene_directorios': tiene_directorios,
            'tiene_file': tiene_file,
            'queries_buenas': queries_buenas,
            'queries_malas': queries_malas,
            'maligna': maligna
        }

        return caracteristicas

    def verificar_tld(self, tld):
        self.cur.execute("SELECT type, tld_manager FROM tlds WHERE domain = %s", ('.' + tld,))
        result = self.cur.fetchone()
        return {
            'type': result[0] if result else None,
            'tld_manager': result[1] if result else None
        }

    def obtener_info_whois(self, dominio):
        try:
            w = whois.whois(dominio)
            return {
                'registrar': w.registrar,
                'creation_date': w.creation_date[0] if isinstance(w.creation_date, list) else w.creation_date,
                'expiration_date': w.expiration_date[0] if isinstance(w.expiration_date, list) else w.expiration_date,
                'updated_date': w.updated_date[0] if isinstance(w.updated_date, list) else w.updated_date,
                'status': w.status
            }
        except Exception as e:
            return {
                'registrar': None,
                'creation_date': None,
                'expiration_date': None,
                'updated_date': None,
                'status': None
            }

    def obtener_version_tls(self, url):
        try:
            parsed_url = urlparse(url)
            contexto = ssl.create_default_context()
            with socket.create_connection((parsed_url.netloc, 443)) as sock:
                with contexto.wrap_socket(sock, server_hostname=parsed_url.netloc) as ssock:
                    return ssock.version()
        except Exception as e:
            return None

    def es_url_acortada(self, url):
        self.cur.execute("SELECT url_provider FROM shorturl")
        short_url_providers = [row[0] for row in self.cur.fetchall()]
        url_parseada = urlparse(url)
        dominio = url_parseada.netloc
        return dominio in short_url_providers

    def es_url_redirigida(self, url):
        try:
            respuesta = requests.head(url, allow_redirects=True)
            return respuesta.url != url
        except requests.RequestException as e:
            print(f"Error al comprobar la URL: {e}")
            return False

    def es_url_acortada_combinada(self, url):
        if self.es_url_acortada(url):
            return True
        return self.es_url_redirigida(url)

    def insertar_caracteristicas(self, caracteristicas):
        columns = sql.SQL(', ').join(map(sql.Identifier, caracteristicas.keys()))
        values = sql.SQL(', ').join(sql.Placeholder() * len(caracteristicas))
        insert_query = sql.SQL("INSERT INTO CaracteristicasURL ({}) VALUES ({})").format(
            columns, values
        )
        self.cur.execute(insert_query, list(caracteristicas.values()))
        self.conn.commit()

    def cerrar_conexion(self):
        self.cur.close()
        self.conn.close()

# Configuración de la base de datos
db_config = {
    'dbname': 'url',
    'user': 'postgres',
    'password': '1234',
    'host': 'localhost',
    'port': '5432',
    'options': '-c client_encoding=UTF8'  # Asegurar la codificación UTF-8
}

geoip_db_path = 'D:\\yo\\GeoLite2-City.mmdb'

extractor = ExtractorCaracteristicasURL(db_config, geoip_db_path)

batch_size = 1000
offset = 500096+1030+192+1191
numero = 1
while True:
    extractor.cur.execute(f"SELECT url, maligna FROM tablaurl LIMIT {batch_size} OFFSET {offset}")
    urls = extractor.cur.fetchall()
    
    if not urls:
        break
    
    for url, maligna in urls:
        
        print(url)
        try:
            caracteristicas = extractor.extraer_caracteristicas(url, maligna)
            extractor.insertar_caracteristicas(caracteristicas)
        except Exception as e:
            print(f"Error al procesar la URL {url}: {e}")
        finally:
            numero += 1
            clear_output(wait=True)
            print(numero, flush=True)
    offset += batch_size

extractor.cerrar_conexion()

print("Características de todas las URL extraídas y almacenadas exitosamente.")
\end{lstlisting}

\end{document}
