Radek Hranický, Adam Horák, Jan Polišenský, Petr Pouč, & Ondřej Ondryáš. (2023). Phishing and Benign Domain Dataset (DNS, IP, WHOIS/RDAP, TLS, GeoIP) (1.0) [Data set]. Zenodo. https://doi.org/10.5281/zenodo.8364668

no se puede muy pesado
https://zenodo.org/records/8364668/files/benign_2307.json?download=1




https://www.kaggle.com/datasets/siddharthkumar25/malicious-and-benign-urls
https://www.kaggle.com/datasets/sid321axn/malicious-urls-dataset
|
|




dominios:

https://www.kaggle.com/datasets/aayushah19/dga-or-benign-domain-names?resource=download


QUERIS

https://github.com/faizann24/Fwaf-Machine-Learning-driven-Web-Application-Firewall/blob/master/goodqueries.txt




palabras



short url:

https://github.com/Spamfighter666/Short-URL-Providers-List/blob/master/Short-URL-Providers-list.txt






tengo varias preguntas: que son los tlds? donde se ven en la url?

como se ve la validez del certificado ssl?

que es Codificación en el dominio y Codificación en el camino (path) ?

como se sabe Cantidad de redirecciones?


Comencemos extrayendo las caractaristicas de las url, comencemos con las lexicas:
-longitud
- cantidad de digitos
-cantidad de letras
