\subsection{Anexo: Configuración de PostgreSQL en Windows}

\subsection*{Instalación de PostgreSQL en Windows}
Para instalar PostgreSQL en Windows, se deben seguir los siguientes pasos:

\begin{enumerate}
    \item \textbf{Descarga del Instalador de PostgreSQL}:
    \begin{itemize}
        \item Visitar la página oficial de PostgreSQL: \href{https://www.postgresql.org/download/windows/}{https://www.postgresql.org/download/windows/}
        \item Descargar el instalador de la versión más reciente proporcionada por EnterpriseDB.
    \end{itemize}
    \item \textbf{Ejecución del Instalador}:
    \begin{itemize}
        \item Ejecutar el archivo de instalación descargado.
        \item Seguir las instrucciones del asistente de instalación:
        \begin{itemize}
            \item \textbf{Seleccionar directorio de instalación}: Elegir la ubicación donde se desea instalar PostgreSQL.
            \item \textbf{Seleccionar componentes}: Dejar los componentes por defecto seleccionados (PostgreSQL Server, pgAdmin 4, Stack Builder, etc.).
            \item \textbf{Configurar el superusuario}: Definir una contraseña para el usuario \texttt{postgres} (es importante recordar esta contraseña, ya que será necesaria para administrar PostgreSQL).
            \item \textbf{Seleccionar puerto}: El puerto por defecto es 5432. No es necesario cambiarlo a menos que sea estrictamente necesario.
            \item \textbf{Configuración regional}: Seleccionar la configuración regional adecuada.
        \end{itemize}
    \end{itemize}
    \item \textbf{Finalización de la Instalación}:
    \begin{itemize}
        \item Una vez completada la instalación, el asistente ofrecerá la opción de ejecutar Stack Builder. Esta opción puede desmarcarse por ahora y finalizar la instalación.
    \end{itemize}
\end{enumerate}

\subsection*{Configuración de PostgreSQL}

\subsubsection*{Acceso a la Interfaz de Línea de Comandos}
\begin{itemize}
    \item Abrir el menú de inicio y buscar \textit{SQL Shell (psql)}.
    \item Ejecutarlo y proporcionar la siguiente información cuando se solicite:
    \begin{itemize}
        \item \textbf{Server}:  [localhost]
        \item \textbf{Database}:  [postgres]
        \item \textbf{Port}:  [5432]
        \item \textbf{Username}: [postgres]
        \item \textbf{Password}
    \end{itemize}
\end{itemize}

\subsubsection*{Creación de una Base de Datos y un Usuario}
Una vez en el shell de PostgreSQL, se puede crear una nueva base de datos y un nuevo usuario con los siguientes comandos SQL:

\begin{lstlisting}[language=SQL, caption=Comandos SQL para crear una base de datos y un usuario]
CREATE DATABASE url;
CREATE USER yourusername WITH ENCRYPTED PASSWORD 'yourpassword';
GRANT ALL PRIVILEGES ON DATABASE url_classification TO yourusername;
\end{lstlisting}

\subsection*{Conexión a PostgreSQL desde Python}

\subsubsection*{Instalación de \texttt{psycopg2}}
Abrir una terminal de comandos (CMD) o PowerShell y ejecutar el siguiente comando para instalar el paquete \texttt{psycopg2}:

\begin{lstlisting}[language=bash, caption=Instalación de psycopg2]
pip install psycopg2-binary
\end{lstlisting}

\subsubsection*{Conexión a la Base de Datos desde un Script de Python}
El siguiente script de Python muestra cómo conectarse a la base de datos PostgreSQL y crear una tabla:

\begin{lstlisting}[language=Python, caption=Script de conexión a PostgreSQL y creación de tabla, breaklines=true]
import psycopg2

try:
    conn = psycopg2.connect(
        dbname="url",
        user="postgre",
        password="1234",
        host="localhost",
        port="5432"
    )
    cur = conn.cursor()
    print("Conexión exitosa a la base de datos")

    # Ejemplo de creacion de una tabla
    cur.execute("""
    CREATE TABLE IF NOT EXISTS TablaUrl (
        id SERIAL PRIMARY KEY,
        url TEXT NOT NULL,
        label BOOLEAN NOT NULL,
        domain VARCHAR(255),
        path TEXT,
        query TEXT,
        length INT,
        num_dots INT,
        num_hyphens INT,
        has_https BOOLEAN,
        has_ip BOOLEAN
    )
    """)
    conn.commit()

    # Cerrar la conexi\ón
    cur.close()
    conn.close()
    print("Tabla creada y conexion cerrada")

except Exception as e:
    print(f"Error conectando a la base de datos: {e}")
\end{lstlisting}