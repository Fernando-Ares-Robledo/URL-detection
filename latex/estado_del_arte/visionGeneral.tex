\subsection{Visión general}

En esta subsección se presenta una visión general del Estado del Arte en la detección de URLs maliciosas. Se describe la importancia del tema y se mencionan los avances y desafíos en el área.

La detección de URLs maliciosas es un aspecto crucial en la ciberseguridad debido al creciente número de ciberdelitos. En 2022, España registró 374,737 ciberdelitos, de los cuales nueve de cada diez fueron fraudes informáticos (estafas) \autocite{acosta2023espana}. Este dato ilustra la magnitud del problema a nivel nacional.

A nivel global, el phishing, una de las formas más comunes de cibercrimen, ha provocado pérdidas superiores a 1.6 mil millones de dolares en 2013 \autocite{konradt2016phishing}. Esta cifra subraya el impacto económico significativo de las URLs maliciosas.

Se han hecho numerosos avances en la detección de URLs maliciosas mediante el uso de tecnologías modernas. Por ejemplo, se ha investigado el uso de modelos de lenguaje preentrenados, como BERT, para la detección de URLs maliciosas \autocite{liu2023malicious}.

El uso de técnicas de inteligencia artificial y machine learning ha demostrado ser efectivo en la mejora de la detección de amenazas. Algunos estudios relevantes incluyen \autocite{bigdata2013}, \autocite{le2018urlnet}, y \autocite{sahoo2017survey}.

No obstante, los ciberdelincuentes continúan desarrollando técnicas avanzadas de evasión para evitar la detección. Entre estas técnicas se encuentran \autocite{tavares2024evasion}, \autocite{defenseevasion}.

Actualmente, existen varias herramientas y soluciones en el mercado para la detección de URLs maliciosas, como Google Safe Browsing \autocite{googlesafebrowsing}, PhishTank \autocite{phishtank}, y URLhaus \autocite{urlhaus}. Estas herramientas, aunque efectivas, no siempre garantizan una detección del 100\%, lo que destaca la necesidad de mejoras continuas en esta área.
