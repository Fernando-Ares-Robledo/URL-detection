\subsubsection{Etapas}


Las etapas del ataque de Ransomware son:
\bigskip
\bigskip
\bigskip
\bigskip
\bigskip
\bigskip
\bigskip
\bigskip
\bigskip
\bigskip
\bigskip
\begin{figure}[H]
\centering

\smartdiagramset{
    set color list={cyan!20, cyan!30, cyan!40, violet!50, cyan!60}, % Ajuste de colores a tonos de azul
    sequence item border color=teal!80, % Color de los bordes de los ítems
    sequence item text color=black,
    sequence item border size=1.5\pgflinewidth,
    sequence item font size=\scriptsize\sffamily,
    additions={
        additional item shape=rectangle,
        additional item fill color=cyan!10, % Color de relleno de los ítems adicionales
        additional item border color=teal!80, % Color del borde de los ítems adicionales
        additional arrow line width=2pt,
        additional arrow tip=to,
        additional arrow color=blue!80, % Color de las flechas adicionales
        additional item font=\scriptsize\sffamily,
        additional item text width=3.5cm,
      }
}
\smartdiagramadd[sequence diagram]{Acceso inicial,Ejecución, Acción sobre objetivos, Chantaje, Negociación de rescate}{
  above of sequence-item1/{Es la fase donde el ransomware logra infiltrarse en el sistema. Puede ocurrir por diferentes medios, como phishing, explotación de vulnerabilidades, o descarga maliciosa.},
  below of sequence-item2/{Una vez dentro del sistema, el ransomware se ejecuta, comenzando el proceso de cifrado de archivos o bloqueo del acceso al sistema.},
  above of sequence-item3/{El ransomware lleva a cabo su propósito principal, que es cifrar archivos críticos y sistemas para hacerlos inaccesibles para el usuario o administrador.},
  below of sequence-item4/{Esta es la fase de comunicación con la víctima, donde los atacantes informan sobre el cifrado y exigen un pago, generalmente en criptomonedas, a cambio de la clave de descifrado.},
  above of sequence-item5/{En algunos casos, puede haber un intercambio entre los atacantes y la víctima para negociar el monto del rescate, aunque no se recomienda hacer pagos ya que no garantiza la recuperación de los datos y puede incentivar a los criminales a continuar con sus actividades ilícitas.}
}
\smartdiagramconnect{to-}{sequence-item1/additional-module1}
\smartdiagramconnect{to-}{sequence-item2/additional-module2}
\smartdiagramconnect{to-}{sequence-item3/additional-module3}
\smartdiagramconnect{to-}{sequence-item4/additional-module4}
\smartdiagramconnect{to-}{sequence-item5/additional-module5}
\bigskip
\bigskip
\bigskip
\bigskip
\bigskip
\caption{Etapas del funcionamiento de un ataque rasomware \autocite{enisa2022ransomware}.}
\end{figure}
\bigskip

\subsubsection{Vectores de ataque}

\definecolor{sprintcolor}{RGB}{210,220,230}
\definecolor{tableheadcolor2}{HTML}{82B1ED}
\definecolor{rulecolor}{HTML}{23E1E8}
\definecolor{cellcolor}{HTML}{b3d9ff}
\definecolor{tableheadcolor}{HTML}{e5f7ff}
\newcolumntype{L}{>{\hsize=.6\hsize\centering\arraybackslash}X} % Columnas más estrechas
\newcolumntype{k}{>{\hsize=1\hsize\centering\arraybackslash}X} 
\newcolumntype{R}{>{\hsize=1.4\hsize\centering\arraybackslash}X} 
\newcolumntype{Y}{>{\centering\arraybackslash}X}
\tcbset{
  tab2/.style={
    enhanced,
    fonttitle=\bfseries,
    %fontupper=\small\sffamily,
    colback=white!10!white, % Fondo del cuerpo del box
    colframe= tableheadcolor2, % Color del borde del box
    colbacktitle=tableheadcolor2, % Usamos el color personalizado para la cabecera
    coltitle=black,
    center title,
    toprule=rulecolor, % Color de la regla superior
    bottomrule=rulecolor, % Color de la regla inferior
    leftrule=rulecolor, % Color de la regla izquierda
    rightrule=rulecolor
  }
}


\begin{table}[H]
\centering
\begin{tcolorbox}[tab2,tabularx={L||k|R}]
    \small
    \centering
    \cellcolor{tableheadcolor}Vector& \cellcolor{tableheadcolor} Tipos & \cellcolor{tableheadcolor} Descripcion\\
    \hline
    \hline
    Phising& 
       
    Correo o SMS
    Navegación web
    Aplicaciones web, portales corporativos, app o intranet y redes sociales
        
    & Este vector incluye métodos engañosos para obtener información sensible de los usuarios, como sus credenciales, a través de comunicaciones que parecen ser de confianza. Los atacantes se hacen pasar por entidades legítimas y utilizan la ingeniería social para inducir a los usuarios a realizar acciones que comprometan su seguridad.  \\
    \hline
    Explotación de Vulnerabilidades & 
        
    Navegación Web
    Endpoints, terminales y dispositivos IoT
    Aplicaciones web, portales corporativos, app o intranet y redes sociales
    Sistemas y elementos de red
        
    & Este vector se centra en el aprovechamiento de debilidades técnicas en software, hardware y configuraciones de seguridad. Los atacantes buscan y explotan fallos de seguridad para obtener acceso no autorizado o causar otros daños.\\
    \hline
    Abuso de Credenciales & 
        
    Uso de contraseñas débiles o por defecto
    Contraseñas comprometidas
    Insider/Sobornos
    Carencias del cifrado
    Debilidades de la cadena de suministro
    &Implica la utilización indebida de información de acceso de usuarios, ya sea obtenida a través de técnicas como el keylogging, fuerza bruta, o ingeniería social, o bien debido a negligencias como el uso de contraseñas predeterminadas o la falta de gestión adecuada de las políticas de cifrado.\\
\end{tcolorbox}
\caption{Vectores más comunes de los ataques rasonware \autocite{incibe2023vectores}.}
\end{table}