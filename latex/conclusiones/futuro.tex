\subsection{Trabajos Futuros}

En el futuro, se planean diversas mejoras y expansiones al sistema desarrollado para la detección de URLs maliciosas. A continuación, se detallan algunas de las direcciones de investigación y desarrollo que se consideran más prometedoras:

\begin{itemize}
    \item \textbf{Recopilación continua de URLs maliciosas:} Es esencial continuar recopilando URLs maliciosas y extrayendo sus características antes de que desaparezcan. Esto permitirá mantener actualizada la base de datos y mejorar continuamente el rendimiento de los modelos de machine learning.

    \item \textbf{Entrenamiento de mejores modelos:} Se planea entrenar modelos más avanzados que consideren una mayor cantidad de características, incluyendo tanto aspectos léxicos como características internas del diseño de la web. Esto incluye analizar qué hacen los scripts, si los textos incitan a realizar acciones fraudulentas, entre otros.

    \item \textbf{Publicación y creación de comunidad:} Se prevé publicar el plugin y el dashboard, así como crear una comunidad de usuarios que utilicen estas herramientas y enriquezcan la base de datos con nuevas URLs maliciosas. La retroalimentación y la colaboración con la comunidad serán clave para mejorar el sistema.

    \item \textbf{Modelo complementario de procesamiento de lenguaje natural:} Se propone complementar el modelo de machine learning actual con un modelo de procesamiento de lenguaje natural (NLP) para analizar el contenido de las páginas web. Esto permitirá detectar si los textos incitan a estafas u otras actividades maliciosas.

    \item \textbf{Colaboraciones con otras entidades:} Se busca establecer colaboraciones con otras entidades, como \textit{urlhaus}, para compartir información y mejorar la detección de URLs maliciosas. Estas colaboraciones pueden enriquecer la base de datos y proporcionar nuevas perspectivas y enfoques para la detección de amenazas.

    \item \textbf{Evaluación y pruebas adicionales:} Se realizarán pruebas adicionales para validar y mejorar el sistema. Esto incluirá el uso de nuevos conjuntos de datos y la evaluación en escenarios específicos para asegurar la robustez y precisión del sistema.

    \item \textbf{Publicación y difusión de resultados:} Se planea publicar los resultados de esta investigación en conferencias y revistas especializadas. Además, se buscarán formas efectivas de difundir y compartir los hallazgos con la comunidad académica y profesional.

\end{itemize}